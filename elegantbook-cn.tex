\documentclass[lang=cn,10pt]{elegantbook}

\title{线性代数笔记:基于 \LaTeX{} 的个人知识总结}
\subtitle{Linear Algebra: Based on \LaTeX{} }

\author{彭正萧 \& PENG Zhengxiao}
\institute{西北农林科技大学}
\date{始于2023年11月20日}
\version{中国农业出版社}
\bioinfo{模板}{\href{https://github.com/ElegantLaTeX/ElegantNote}{ElegantNote}}

\extrainfo{人不能像走兽那样活着,应该追求知识和美德。 ——但丁}

\setcounter{tocdepth}{3}

\logo{logo.pdf}
\cover{cover.jpg}

% 本文档命令
\usepackage{array}
\newcommand{\ccr}[1]{\makecell{{\color{#1}\rule{1cm}{1cm}}}}
\newcommand{\matA}{\mathbf{A}}

% 修改标题页的橙色带
% \definecolor{customcolor}{RGB}{32,178,170}
% \colorlet{coverlinecolor}{customcolor}

\begin{document}

\maketitle
\frontmatter

\tableofcontents

\mainmatter

\chapter{矩阵}

	\section{矩阵的概念}
	矩阵的概念
	\section{矩阵的运算}
	矩阵的运算
	\section{分块矩阵}
	你是谁啊,我不知道啊
	
\chapter{\( n \)阶矩阵的行列式}

	\section{\( n \)阶行列式的概念}
	
	\section{行列式的性质}
	
	\section{\( n \)阶行列式的计算}
	
\chapter{\( n \)阶矩阵的逆与矩阵的秩}

	\section{\( n \)阶矩阵的逆}
	
	\section{矩阵的初等变换}
	
	\section{初等矩阵与求\( n \)阶矩阵的逆}
	
	\section{矩阵的秩}
	
\chapter{线性方程组与向量组的秩}

	\section{线性方程组的消元法与解的存在性}
		\subsection{线性方程组解的存在性}
		\begin{enumerate}
			\item 非齐次线性方程组有解的充分必要条件
			\begin{theorem}
				线性方程组\( \matA x = b \)有解的充分必要条件是\( R(\matA) = R(\bar{A}) \). \\
				当\( R(A) = R(\bar{A}) = n \)(\( n \)为方程组中未知数的个数)时,线性方程组有唯一解. \\
				当\( R(A) = R(\bar{A}) < n \)(\( n \)为方程组中未知数的个数)时,线性方程组有无穷多个解.
			\end{theorem}

		\end{enumerate}
	\section{向量组与矩阵}
	
	\section{向量组的线性相关性}
	
	\section{向量组的秩}
	
	\section{线性方程解的结构}
	
\chapter{\( n \)阶矩阵的对角化与二次型}

	\section{向量的内积与正交矩阵}
	
	\section{矩阵的特征值和特征向量}
	
	\section{矩阵的对角化}
	
	\section{实对称矩阵的对角化}
	
	\section{二次型及其标准型}
	
	\section{二次型的正定性}

\chapter{线性空间}

	\section{线性空间的概念}
	
	\section{满秩坐标变换}
	
	\section{线性变换}

\end{document}
