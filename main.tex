\documentclass[lang=cn,10pt]{elegantbook}

\title{线性代数:基于 \LaTeX{} 的个人知识总结}
\subtitle{Linear Algebra: Based on \LaTeX{} }

\author{彭正萧 \& PENG Zhengxiao}
\institute{西北农林科技大学}
\date{始于2023年11月20日}
\version{中国农业出版社}
\bioinfo{模板}{\href{https://github.com/ElegantLaTeX/ElegantNote}{ElegantNote}}

\extrainfo{人不能像走兽那样活着,应该追求知识和美德。 ——但丁}

\setcounter{tocdepth}{3}

\logo{logo.pdf}
\cover{cover.jpg}

% 本文档命令
\usepackage{array}
\newcommand{\ccr}[1]{\makecell{{\color{#1}\rule{1cm}{1cm}}}}

\newcommand{\matA}{\mathbf{A}}

% 修改标题页的橙色带
% \definecolor{customcolor}{RGB}{32,178,170}
% \colorlet{coverlinecolor}{customcolor}

\begin{document}

\maketitle
\frontmatter

\tableofcontents

\mainmatter

\chapter{矩阵}

	\section{矩阵的概念}
	\subsection{矩阵的概念}
	\begin{definition}
		由\( m \times n \)个数\( a_{ij} (i = 1, 2, \cdots, m; j = 1, 2, \cdots, n) \)排成\( m \)行\( n \)列的矩形数表
		\[
			\begin{bmatrix}
				a_{11} & a_{12} & \cdots & a_{1n} \\
				a_{21} & a_{22} & \cdots & a_{2n} \\
				\vdots & \vdots & \ddots & \vdots \\
				a_{m1} & a_{m2} & \cdots & a_{mn} \\
			\end{bmatrix}
		\]
		此数表称做\( m \)行\( n \)列{\heiti 矩阵},简称\( m \times n \){\heiti 矩阵},其中\( a_{ij} \)称为矩阵的第\( i \)行第\( j \)列的{\heiti 元素},记为
		\[
			\matA = 
			\begin{bmatrix}
				a_{11} & a_{12} & \cdots & a_{1n} \\
				a_{21} & a_{22} & \cdots & a_{2n} \\
				\vdots & \vdots & \ddots & \vdots \\
				a_{m1} & a_{m2} & \cdots & a_{mn} \\
			\end{bmatrix}
		\]
		有时也记为\( \matA = [a_{ij}]_{m \times n} \)或\( \matA = [a_{ij}] \)或\( A_{m \times n} \).
	\end{definition}
	
	\begin{remark}
		\begin{itemize}
			\item 矩阵一般都用黑体大写字母\( \matA, \mathbf{B}, \mathbf{C}, \cdots \)来表示.
			\item 矩阵既可以用方括号表示,又可以用圆括号表示,即\\
				\(
					\begin{bmatrix}
						a_{11} & a_{12} & \cdots & a_{1n} \\
						a_{21} & a_{22} & \cdots & a_{2n} \\
						\vdots & \vdots & \ddots & \vdots \\
						a_{m1} & a_{m2} & \cdots & a_{mn} \\
					\end{bmatrix}
					\Leftrightarrow
					\begin{pmatrix}
						a_{11} & a_{12} & \cdots & a_{1n} \\
						a_{21} & a_{22} & \cdots & a_{2n} \\
						\vdots & \vdots & \ddots & \vdots \\
						a_{m1} & a_{m2} & \cdots & a_{mn} \\
					\end{pmatrix}
				\)
			\item 如果矩阵\( \matA \)的元素\( a_{ij} \)全为实数,就称\( \matA \)实矩阵.
			\item 如果矩阵\( \matA \)有一个元素\( a_{ij} \)为复数,就称\( \matA \)为复矩阵.
		\end{itemize}
	\end{remark}
	
	\subsection{几种特殊的矩阵}
		\begin{enumerate}
			\item 对角矩阵
			\item 上(下)三角矩阵
			\item 对称矩阵与反对称矩阵
		\end{enumerate}
	\section{矩阵的运算}
	矩阵的运算
	\section{分块矩阵}
	你是谁啊,我不知道啊
	
\chapter{\( n \)阶矩阵的行列式}

	\section{\( n \)阶行列式的概念}
	
	\section{行列式的性质}
	
	\section{\( n \)阶行列式的计算}
	
\chapter{\( n \)阶矩阵的逆与矩阵的秩}

	\section{\( n \)阶矩阵的逆}
	
	\section{矩阵的初等变换}
	
	\section{初等矩阵与求\( n \)阶矩阵的逆}
	
	\section{矩阵的秩}
	
\chapter{线性方程组与向量组的秩}

	\section{线性方程组的消元法与解的存在性}
		\subsection{线性方程组解的存在性}
		\begin{enumerate}
			\item 非齐次线性方程组有解的充分必要条件
			\begin{theorem}
				线性方程组\( \matA x = b \)有解的充分必要条件是\( R(\matA) = R(\bar{A}) \). \\
				当\( R(A) = R(\bar{A}) = n \)(\( n \)为方程组中未知数的个数)时,线性方程组有唯一解. \\
				当\( R(A) = R(\bar{A}) < n \)(\( n \)为方程组中未知数的个数)时,线性方程组有无穷多个解.
			\end{theorem}

		\end{enumerate}
	\section{向量组与矩阵}
	
	\section{向量组的线性相关性}
	
	\section{向量组的秩}
	
	\section{线性方程解的结构}
	
\chapter{\( n \)阶矩阵的对角化与二次型}

	\section{向量的内积与正交矩阵}
	
	\section{矩阵的特征值和特征向量}
	
	\section{矩阵的对角化}
	
	\section{实对称矩阵的对角化}
	
	\section{二次型及其标准型}
	
	\section{二次型的正定性}

\chapter{线性空间}

	\section{线性空间的概念}
	
	\section{满秩坐标变换}
	
	\section{线性变换}

\chapter{你是谁}

%\printbibliography[heading=bibintoc, title=\ebibname]
\appendix


\end{document}
